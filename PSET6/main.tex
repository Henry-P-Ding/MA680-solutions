\documentclass[12pt]{extarticle}
\usepackage{amsmath}
\usepackage{amsthm}
\usepackage{amssymb}
\usepackage{enumitem}
\setlist{listparindent=\parindent, parsep=0pt,}

\author{Henry Ding}

% xcolor package
\usepackage[usenames,dvipsnames]{xcolor}

% ams packages
\usepackage{amsthm}
\usepackage{amssymb}
\usepackage{amsmath}

% physics package
\usepackage{physics}

% chemistry package
\usepackage[version=4]{mhchem}

% siunitx package
\usepackage{siunitx}
\AtBeginDocument{\RenewCommandCopy\qty\SI}
\DeclareSIUnit{\degreeFahrenheit}{\unit{\degree}F}
\DeclareSIUnit{\degreeRankine}{\unit{\degree}R}
\DeclareSIUnit{\atm}{atm}

% tikz packages
\usepackage{tikz}

% Document style
\newlength{\alphabet}
\settowidth{\alphabet}{\normalfont abcdefghijklmnopqrstuvwxyz}
\usepackage[textwidth=2.5\alphabet]{geometry}
\frenchspacing
\usepackage[protrusion=true, expansion=true]{microtype}

% subfiles and standalone packages
\usepackage{standalone}
\usepackage{subfiles} % best loaded last in the preamble

% Theorems
\usepackage{thmtools}
\usepackage[framemethod=TikZ]{mdframed}
\mdfsetup{skipabove=1em,skipbelow=0em}

% source: https://github.com/gillescastel/lecture-notes/blob/master/algebraic-topology/preamble.tex
\declaretheoremstyle[
	headfont=\bfseries\sffamily\color{ForestGreen!70!black}, bodyfont=\normalfont,
	mdframed={linewidth=2pt, rightline=false, topline=false, bottomline=false, linecolor=ForestGreen, backgroundcolor=ForestGreen!5,}
]{thmgreenbox}

% source: https://github.com/gillescastel/lecture-notes/blob/master/algebraic-topology/preamble.tex
\declaretheoremstyle[
	headfont=\bfseries\sffamily\color{NavyBlue!70!black}, bodyfont=\normalfont,
	mdframed={linewidth=2pt, rightline=false, topline=false, bottomline=false, linecolor=NavyBlue, backgroundcolor=NavyBlue!5,}
]{thmbluebox}

\declaretheorem[style=thmgreenbox, name=Problem]{prob}
\declaretheorem[style=thmbluebox, name=Solution]{sol}

% differential operator
\newcommand{\diff}{\ensuremath{\operatorname{d}\!}}

\title{PSET6}
\author{Henry Ding}
\date{April 2024}

\begin{document}

\maketitle

\begin{prob} \label{prob:problem_1}
	A simple question:
	\begin{enumerate}[label=(\alph*)]
		\item Show that if G is a simple graph, then $\varepsilon \leq \binom{v}{2}$
		\item Show that $\varepsilon = \binom{v}{2}$ if and only if $G$ is also complete (assuming it is simple).
	\end{enumerate}
\end{prob}

\begin{sol}
	A simple solution:
	\begin{enumerate}[label=(\alph*)]
		\item
		      Since $G$ is a simple graph, there is at most one edge between two vertices and no edge connects a vertex to itself.
		      Therefore, for any pair of distinct vertices, there is at most one edge connecting them, and all edges must connect two distinct vertices.
		      The number of pairs of distinct vertices is $\binom{v}{2}$, so
		      \begin{equation}
			      \varepsilon \leq \binom{v}{2}.
		      \end{equation}
		\item We aim to prove both directions of the biconditional. Note that $G$ is a simple graph.
		      \begin{enumerate}[label=\arabic*.]
			      \item We aim to show that if $G$ is complete, then $\varepsilon = \binom{v}{2}$.
			            Because $G$ is complete, then each vertex is connect by an edge to every other vertex.
			            Because $G$ is also simple, we also know that each vertex is connected by exactly \emph{one} to every other vertex.
			            The number of edges then equals the number of pairs of distinct vertices, which is given by $\binom{v}{2}$.
			            Therefore,
			            \begin{equation}
				            \varepsilon = \binom{v}{2}
			            \end{equation}
			      \item We aim to show that if $\varepsilon = \binom{v}{2}$, then $G$ is complete.
			            First, note that if all of those unconnected vertex pairs are connected by exactly one edge, then the number of edges $\varepsilon = \binom{v}{2}$.
			            Suppose towards contradiction that there exist a some pairs of vertices that have no edge connecting them but all pairs of vertices are connected by at most one edge(so $G$ is simple but not complete).
			            Then, the number of edges must be less than if all of those unconnected vertex pairs were connected by exactly one edge, or $\varepsilon < \binom{v}{2}$.
			            However, this contradicts the assumption that $\varepsilon = \binom{v}{2}$, so there does not exist any pairs of vertices that have no edge connecting them.
		      \end{enumerate}
	\end{enumerate}
\end{sol}

\begin{prob} \label{prob:problem_2}
	Recall that the \textit{degree} of a vertex $d(v)$ in $G$ is the number of edges incident with $v$.
	\begin{enumerate}[label=(\alph*)]
		\item Explain why \begin{equation}
			      \sum_{v\in V}d(v) = 2\varepsilon
		      \end{equation}
		\item Prove that, in any graph, there is an even number of vertices of odd degree.
		\item Show that in a simple graph (with $|V| > 1$), there are at least two vertices with the same degree.
	\end{enumerate}
\end{prob}

\begin{sol} \hfill
	\begin{enumerate}[label=(\alph*)]
		\item To count the number of edges in a graph, we can start by adding together the degree of all vertices, which can be written as the expression
		      \begin{equation}
			      \sum_{v\in V} d(v). \label{eq:sum_vertex_degrees}
		      \end{equation}
		      However, some pairs of vertices must be adjacent, so Eq.~\eqref{eq:sum_vertex_degrees} over counts the number of edges as there are some duplicate edges in the count.
		      Every edge is connected to two vertices (or a single vertex twice), so Eq.~\eqref{eq:sum_vertex_degrees} exactly double counts the number of edges.
		      Therefore,
		      \begin{equation}
			      \sum_{v\in V}d(v) = 2 \varepsilon. \label{eq:p2_vertex_edge_relation}
		      \end{equation}

		      Note that in the case where an edge connects a vertex $x_1$ to itself in a loop, the edge contributes $2$ to $d(x_1)$, so Eq.~\eqref{eq:sum_vertex_degrees} would still double count the edge.
		\item Note that Eq.~\eqref{eq:p2_vertex_edge_relation} implies that the expression
		      \begin{equation}
			      \sum_{v\in V} d(v) = 2 \varepsilon
		      \end{equation}
		      is even, since $\varepsilon \in \mathbb{Z}^+$.
		      Also, note that the number of vertices of odd degree is either odd or even.

		      Suppose towards contradiction that there is an odd number of vertices of odd degree in $G$.
		      Define two sets
		      \begin{align*}
			      O & = \{x \in V : 2 \nmid d(x) \} \\
			      E & = \{x \in V : 2 \mid d(x) \}
		      \end{align*}
		      Note that $d(v)$ is either even or odd but not both, so $V = O \cup E$ and $O \cap E = \varnothing$.
		      Therefore,
		      \begin{equation}
			      \sum_{v\in V} d(v) = \sum_{x_O \in O} d(x_O) + \sum_{x_E \in E} d(x_E). \label{eq:p2_odd_even_sum}
		      \end{equation}
		      Note that every addend in the first term of Eq.~\eqref{eq:p2_odd_even_sum} is odd and there are an odd number of addends, so the first term is odd.
		      Every addend in the second term of Ed.~\eqref{eq:p2_odd_even_sum} is even, so the second term is even.
		      Therefore, both sides of Eq.~\eqref{eq:p2_odd_even_sum} is odd.
		      However, this contradicts the Eq.~\eqref{eq:p2_vertex_edge_relation} result from part (a) that
		      \begin{equation*}
			      \sum_{v\in V} d(v)
		      \end{equation*}
		      is even.
		      Therefore, there is an even number vertices of odd degree in $G$.
		\item First, note that for any vertex $v$, $d(v) \in \mathbb{N}$.
		      There are two cases:
		      \begin{enumerate} [label=\arabic*.]
			      \item Suppose some vertex connects with every other vertex by an edge so that its degree is $|V| - 1$.
			            Then, there are no vertices that do not connect to any other vertices, so there are no vertices with degree $0$.
			            Because the graph is simple, a vertex can only be connected by a single edge to a different vertex.
			            For any single vertex, there are only $|V-1|$ other vertices, so the maximum degree of a vertex is $|V| - 1$.
			            Therefore, the degrees of any vertices $v$ must fall in the interval
			            \begin{equation}
				            1 \leq d(v) \leq |V| - 1, \label{eq:p2_case1_range}
			            \end{equation}
			            which corresponds to $|V| - 1$ distinct values for the degree of a vertex.
			            There are $|V|$ vertices, so by the Pigeon Hole Principle, there are at least two vertices with the same degree.

			            Note that $|V| > 1$, so Eq.~\eqref{eq:p2_case1_range} always leaves least one distinct value for $d(v)$ that we can apply the Pigeon Hole Principle to.
			      \item Suppose there is no vertex that connects with every other vertex by an edge.
			            Because the graph is simple, a vertex can only be connected by a single edge to a different vertex.
			            For any single vertex, there are only $|V|-1$ other vertices, so the maximum degree of a vertex is $|V| - 1$.
			            Then, in this case, there are no vertices with degree $|V| - 1$ or greater.
			            Therefore, the degrees of any vertices $v$ must fall in the interval
			            \begin{equation}
				            0 \leq d(v) \leq |V| - 2, \label{eq:p2_case2_range}
			            \end{equation}
			            which corresponds to $|V| - 1$ distinct values for the degree of a vertex.
			            There are $|V|$ vertices, so by the Pigeon Hole Principle, there are at least two vertices with the same degree.

			            Note that $|V| > 1$, so Eq.~\eqref{eq:p2_case2_range} always leaves least one distinct value for the degree of a vertex.
		      \end{enumerate}
	\end{enumerate}
\end{sol}


\begin{prob} \label{prob:problem_3}
	Prove that in any group (of two or more people), there must be two people with the same number of friends in that group. Assume that friendship is symmetric.
\end{prob}

\begin{sol}
	Create a graph $G$ such that each person is assigned a vertex. Also, if two distinct people are friends, then the vertices assigned to those two people is connected by an edge.

	Note that two people are either friends or not friends, so there is at maximum one edge connecting two vertices in $G$.
	Also, note that in defining $G$ earlier, an edge can only connect two distinct vertices for two distinct people.
	Therefore, as defined, $G$ is a simple graph.
	Also, $G$ has at least two vertices, since the group has two or more people.

	Because $G$ is simple and has at least two vertices, the result from Problem~\ref{prob:problem_2}c tells us that there are two vertices with the same degree.
	Therefore, there are two people with the same number of friends.
\end{sol}

\begin{prob} \label{prob:problem_4}
	Let $S = \{x_1, x_2, \ldots, x_n\}$ be a set of points in the coordinate plane such that $|x_i - x_j| \geq 1$ for $i \neq j$. Show that there are at most $3n$ pairs of points whose distance is exactly $1$.
\end{prob}

\begin{sol} \label{sol:solution_4}
	Define a graph $G$ such that every element of $S$ corresponds to a vertex and two points who are exactly distance $1$ apart are connected by exactly one edge.

	First, we aim to show that for any given vertex in $G$, its maximum degree is $6$.
	Consider an arbitrary point $x_\alpha$ corresponding to vertex $v_\alpha$.
	Any adjacent vertex to $v_\alpha$ in must correspond to a point on a circle $O$ of radius $1$ centered on $x_\alpha$.
	Suppose towards contradiction that $n > 6$ points from $S$ are on circle $O$.
	Consider the $n$ arcs joining two adjacent points (I'm \emph{not} referring to vertex adjacency here, just two points that have no other point in between them on $O$.)

	Next, suppose towards contradiction (again) that none of the $n$ arcs have a central angle less than $\pi / 3$.
	Then, the sum of the central angles of the $n$ arcs must sum to greater than $2\pi$ because $n > 6$.
	However, the $n$ arcs form a circle together, meaning the sum of their central angles must equal $2\pi$.
	Therefore, there must exist one of the $n$ arcs with a central angle less than $\pi / 3$ between some points $x_\beta$ and $x_\gamma$ by contradiction.

	Now that we proven that they exist, consider the distinct points $x_\beta$ and $x_\gamma$ in $S$ and on Circle $O$, where the minor arc formed $x_\beta, x_\gamma$ subtends a central angle $0 < \theta < \pi / 3$.
	Form a triangle with points $x_\alpha$, $x_\beta$, and $x_\gamma$.
	The Law of Cosines gives the distance $d_{\beta\gamma}$ between $x_\beta, x_\gamma$ as
	\begin{align}
		d_{\beta\gamma} & = 1^2 + 1^2 - 2 \cdot 1 \cdot 1 \cdot \cos(\theta) \\
		                & = 2 - 2\cos(\theta)                                \\
		                & < 2 - 2\cos(\pi / 3)                               \\
		                & < 1, \label{eq:p4_circle_point_distance}
	\end{align}
	where we have used the facts that the distance between $x_\alpha$ and $x_\beta$ is $1$ because $x_\beta$ is on $O$, the distance between $x_\alpha$ and $x_\gamma$ is $1$ because $x_\gamma$ is on $O$, and $\cos(\theta) > \cos(\pi / 3)$ for $0 < \theta < \pi / 3$.
	However, Eq.~\eqref{eq:p4_circle_point_distance} contradicts the assumption that any two distinct points in $S$, including $x_\beta, x_\gamma$, have a distance greater than or equal to $1$.
	Therefore, there can not exist more than $6$ points from $S$ on circle $O$ by contradiction.

	Next, we aim to show that the maximum number of edges in $G$ is $3n$.
	Suppose towards contradiction that number of edges $\varepsilon$ in $G$ is greater than $3n$.
	Then, Eq.~\eqref{eq:p2_vertex_edge_relation} implies
	\begin{equation}
		\sum_{v_i}^n d(v_i) > 6n
	\end{equation}
	where $v_i$ is an indexed vertex of $G$ and $d(v)$ is the degree of vertex $v_i$.
	By the Pigeon Hole Principle, there must exist some vertex $v_0$ such that $d(v_0) > 6$.
	However, we have previously shown that the maximum number of edges of $G$ is $6$.
	Therefore, the number of edges $\varepsilon$ is at most $3n$ by contradiction.

	From our definition of $G$, each edge corresponds to a single pair of two points that are exactly distance $1$ apart.
	Because $\varepsilon \leq 3n$, the number of pairs of two points distance $1$ apart is at most $3n$.
\end{sol}

\end{document}

